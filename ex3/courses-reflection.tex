\documentclass[a4paper]{article}
\begin{document}

\tableofcontents
\newpage
\section{Document Introduction}
This will be a brief document reflecting upon my studies so far at the University of Manchester, looking at each course unit
individually. I will give a brief outline of each course unit and then give three points I like and three points I dislike
about the course unit.

\section{COMP10120 - Team Project}
This course unit will be a key part of this year, finishing with a team project at the end of the year. This project will
require the use of knowledge from other course units as well as the work we do in 10120.
\subsection{Things I like...}
\begin{enumerate}
\item Helps build team work
\item Good way to meet new people at the start of the year
\item Uses elements of all the other course units
\end{enumerate}
\subsection{Things I don't like...}
\begin{enumerate}
\item Might be difficult to work with some members of the team
\item Involves researching vague topics
\item May take a large chunk of time at the end of the year
\end{enumerate}

\section{COMP11120 - Mathematical Techniques}
With maths being a fundamental skill in computer science, it is no surprise one of the course units in the first year focuses on
mathematical skills. All these skills relate to some element of computer science and will prove useful in the next few years.
\subsection{Things I like...}
\begin{enumerate}
\item Gives you skills you need to complete tasks in other units
\item Good lecturers that deliver new concepts clearly
\item Covers areas of maths that I haven't done previously
\end{enumerate}
\subsection{Things I don't like...}
\begin{enumerate}
\item Lectures that cover areas of maths I've already learnt feel repetitive
\item The language used in the notes is confusing at first
\item New areas of maths require alot of effort to learn properly
\end{enumerate}

\section{COMP12111 - Computer Engineering}
A course unit that focuses on the fundamentals of computer engineering, COMP12111 gives you the knowledge you need to understand how
logic circuits are made and the importance they have in the field.
\subsection{Things I like...}
\begin{enumerate}
\item The lecturer makes the lectures interesting
\item The use of quizzes makes the lectures interactive
\item The topics lead on naturally from one to another
\end{enumerate}
\subsection{Things I don't like...}
\begin{enumerate}
\item Walking over to another building for some lectures
\item You often miss the start of lectures because of the walk and previous examples classes
\item It may be difficult to pick up some of the abstract concepts within the lecture
\end{enumerate}

\section{COMP15111 - Computer Architecture}
Focusing on the fundamentals of how computers work, COMP15111 gives you a good understanding of how the processor and memory work and
interact. This basic knowledge is then built on by the other course units.
\subsection{Things I like...}
\begin{enumerate}
\item Provides the knowledge that everything builds off
\item The lectures make you view computers in a different way
\item The lecturer makes the boring topics interesting
\end{enumerate}
\subsection{Things I don't like...}
\begin{enumerate}
\item Alot of the topics are repetitive and therefore boring
\item The microphone in the lecture theatre seems quiet for the lectures
\item It is difficult to see the relevance of some of the topics
\end{enumerate}

\section{COMP16121 - Java}
Obviously one of the courses for computer science focuses on coding, in particular Java. This course unit gives you the skills that can be 
transferred to any high level programming language.
\subsection{Things I like...}
\begin{enumerate}
\item The lecturer's shirts
\item The slides are thought provoking
\item Makes you aware of the habits that you may have built up over time
\end{enumerate}
\subsection{Things I don't like...}
\begin{enumerate}
\item Some of the basics may seem boring to people who have programmed in the past
\item It will be difficult to drop some of the bad coding habits
\item Over documenting code that seems easy (but I see why we have to do it)
\end{enumerate}


\end{document}
